\documentclass{amsart}
%\usepackage{showkeys}
%\documentclass[12pt]{iopart}

\usepackage{amssymb,amsfonts}
\usepackage{color}
%\usepackage{showlabels}

\setlength{\oddsidemargin}{-0.4cm}
\setlength{\evensidemargin}{-0.4cm}
\setlength{\topmargin}{-1.5cm}
\setlength{\textwidth}{17.5cm}
\setlength{\textheight}{24cm}


\newtheorem{theorem}{Theorem}[section]
\newtheorem{lemma}[theorem]{Lemma}
\newtheorem{proposition}[theorem]{Proposition}
\newtheorem{Conjecture}[theorem]{Conjecture}
\newtheorem{assumption}{Assumption}
\newtheorem{corollary}[theorem]{Corollary}
\newtheorem{definition}[theorem]{Definition}
\newtheorem{Algorithm}{Algorithm}[section]
\newtheorem{example}[theorem]{Example}
\newtheorem{remark}[theorem]{Remark}
\newtheorem{conj}[theorem]{Conjecture}
\newtheorem{Lemma}[theorem]{Lemma}
\newtheorem{Corollary}[theorem]{Corollary}
\newtheorem{Definition}[theorem]{Definition}
\newtheorem{Example}[theorem]{Example}
\newtheorem{Notation}{Notation}
\newtheorem{Theorem}[theorem]{Theorem}
\newtheorem{Remark}[theorem]{Remark}
\newtheorem{Proposition}{Proposition}
\newcommand{\be}{\begin{equation}}
\newcommand{\ee}{\end{equation}}
\newcommand{\beq}{\begin{equation*}}
\newcommand{\eeq}{\end{equation*}}
\newcommand{\ben}{\begin{eqnarray}}
\newcommand{\een}{\end{eqnarray}}
\newcommand{\bea}{\begin{eqnarray*}}
\newcommand{\eea}{\end{eqnarray*}}
\newcommand{\s}{ {\mathcal{x}}}
\newcommand{\At}{ {\widetilde{A}}}
\def\Gammat{{\widetilde{\Gamma}}}
\def\gammat{{\widetilde{\gamma}}}
\def\wt{\tilde{w}}
\newcommand{\Hc}{ {\mathcal{H}}}
\def\cK{{\mathcal K}}
\def\cH{{\mathcal H}}
\newcommand{\K}{ {\mathcal{K}}}
\newcommand{\cL}{ {\mathcal{L}}}
\newcommand{\Sc}{ {\mathcal{S}}}
\newcommand{\Sct}{ {\widetilde{\mathcal{S}}}}
\newcommand{\Sco}{ {\overline{\mathcal{S}}}}
\newcommand{\Scto}{ {\overline{\widetilde{\mathcal{S}}}}}
\newcommand{\Tc}{ {\mathcal{T}}}
\newcommand{\sign}{\mbox{\rm sign}}
\def\ker{{\mathrm{ker\,}}}
\def\Ran{{\mathrm{Ran\,}}}
\def\Span{{\mathrm{Span\,}}}
\newcommand{\Rr}{{\mathbb{R}}}
\newcommand{\Nn}{{\mathbb{N}}}
\newcommand{\Cc}{{\mathbb{C}}}
\newcommand{\Zz}{{\mathbb{Z}}}
\newcommand{\lek}{ \Lambda_{\eta,K}}
\newcommand{\Eta}{\eta}
\newcommand{\Zeta}{\zeta}
\newcommand{\llangle}{\left\langle}
\newcommand{\rrangle}{\right\rangle}
\renewcommand{\ll}{\left\langle}
\newcommand{\rr}{\right\rangle}
\newcommand{\eps}{\varepsilon}
\newcommand{\A}{A}
\def\C{\mathbb C}
\def\R{\mathbb R}
\def\N{\mathbb N}
\def\Z{\mathbb Z}
\def\CD{ D}
\newcommand{\fH}{{\frak H}}
\newcommand{\supp}{\mbox{\rm supp}}
\newcommand{\essran}{\mbox{\rm essran}}

\newcommand{\norm}[1]{\left\Vert#1\right\Vert}
\newcommand{\trinorm}[1]{\left\vert\!\vert\! \vert#1\vert\!\vert\!\right\vert}

\newcommand{\twovec}[2]{\left(\begin{array}{c} #1 \\  #2 \end{array}\right)}

\begin{document}
The first LMS Bath-WIMCS Analysis Day took place in Swansea on Friday 20th March:
\[ \mbox{\tt
http://math.swansea.ac.uk/staff/vm/LMS-WIMCS-Bath-2015/} \]
Our intention was that each meeting in this series should involve a selection of topics from those
proposed in the application, to encourage the widest possible attendance. In fact all three topics
(calculus of variations and nonlinear PDEs; asymptotics, homogenisation and applications; spectral
theory and related topics) were represented in this meeting, although a common theme emerged
around homogenisation which appeared in different guises in four of the five talks.
\vspace{2mm}

\noindent {\bf Federica Dragoni (Cardiff)} spoke on `Stochastic homogenisation for Hamilton-Jacobi equations'.
She reviewed two 1999 papers (of Souganidis and of Rezakhanlou and Tarver) which deal with
the equation
\[ u_t^\varepsilon + H(x/\varepsilon,Du,\omega) = 0, \;\;\; x\in {\mathbb R}^N, \]
in which $\omega$ is a random variable and the Hamiltonian $H$ satisfies a special Lipschitz condition
with respect to its first argument. It turns out that if $H$ is stationary ergodic then the homogenised problem
is deterministic, in the sense that the viscosity solutions converge to a function which does not depend on
$\omega$.

In the second part of her talk, Dr Dragoni spoke about her own recent work with Mannucci and Marchi,
dealing with stochastic homogenisation for a Hamiltonian of the form $H(x/\varepsilon,\sigma(x)Du, \omega)$
in which $\sigma$ is a matrix of Carnot type. Carnot groups are anisotropic at every scale, so the scale
$x/\varepsilon$ has to be adapted to the new group structure and additional difficulties arise from the
fact that the method of approximating initial conditions by affine functions now fails.
\vspace{2mm}

\noindent {\bf Alexander Kiselev (L'viv, Ukraine)} spoke on `Mathematical modelling of scattering in the presence of
metamaterial objects'. His talk reviewed the models in the literature to date and presented recent joint work with Cherednichenko
in one and two dimensions, in particular on the rigorous construction of scattering matrix in the low-energy limit. Their approach
allows not only a superlens of an arbitrary geometry, but also illuminates the origin and particulars of the process of cloaking.
They showed that the problem of cloaking can be explained in terms of `resonances' for the compact scatterer, whose
spectral properties give rise to cloaking.
\vspace{2mm}

{\bf \noindent Valery Smyshlyaev (UCL) } spoke on `Two-scale homogenisation of a general class of high-contrast PDE systems'.
The first part of this talk was a review suitable for a general audience of mathematical analysts, while the second part presented
joint work with Cooper (formerly Leverhulme-WIMCS Fellow in Cardiff, now in Montpellier, soon to move to Bath) and Kamotski (UCL).
For a general class of high-contrast PDE systems, they show that under a decomposition assumption a two-scale version of the strong
resolvent convergence holds, with a well-defined two-scale limit operator. This implies two-scale convergence of semigroups with
applications to a wide class of micro-resonant time-dependent problems.
\vspace{2mm}

\noindent {\bf Matthew Lettington (Cardiff)} gave our only talk in analytic number theory, on `Higher dimensional interlacing Fibonacci
sequences, continued fractions and Chebyshev polynomials.' He presented joint work with Hindmarsh and Pryce studying higher-dimensional
interlacing Fibonacci sequences and their multi-dimensional continued fractions, generated both from Chebyshev type functions and $m$-dimensional
recurrence relations. For each $m$, there exist both rational and integer versions of these sequences and the underlying $p$-adic structure of the
rational sequence allows recovery of  enables the integer sequence. In particular, for the positive index sequences, one can clear fractions by knowing
the number of prime divisors of $2m + 1$; in the negative index case the `excess' prime factors can be removed using Weisman?s congruence.
When $2m + 1$ is a prime these two processes come into alignment.
From either the rational or the integer sequences, Lettington showed how to construct a continued fraction vector in ${\mathbb  Q}^m$, which converges
to an irrational algebraic point in ${\mathbb R}^m$. The sequence terms can be expressed as simple recurrences, trigonometric sums, binomial
polynomials and as sums over ratios of powers of the diagonals of the regular unit $m$-gon. These sequences exhibit a `rainbow' quality, corresponding
to the Fleck numbers at negative indices and the $m$-dimensional Fibonacci numbers at positive indices. The families of orthogonal generating polynomials
defining the recurrence relations employed turn out to be divisible by the minimal polynomials of certain algebraic numbers, and the three-term recurrences
and differential equations for these polynomials can be found.
\vspace{2mm}

\noindent{\bf Lucia Scardia (Bath)} spoke on `Homogenisation of dislocation dynamics'. Plastic deformation of a metal is caused by the movement of
curve-like defects (called dislocations) in its crystal lattice. It is not known how to use this microscale information to make theoretical predictions at the
macroscopic scale. Nevertheless, macroscopic plasticity is heavily dependent on dynamic properties of the dislocation curves. Mora, Peletier and Scardia
recently upscaled a time-dependent system of discrete, interacting dislocations by combining Gamma-convergence methods with the theory of rate-independent
systems. In the continuum limit they obtain an evolution law for the dislocation density. The talk presented this result and possible extensions to more realistic
complex systems.


\begin{tabular}{ll}
Attendance & Male/Female\\
\hline
Invited speakers        &          3/2\\
Research students (host inst) &    4/2\\
Research students (other inst’s) &  5/0\\
Other participants (host inst) &   6/1\\
Other participants (other inst’s) & 2/0\\
Total                           & 20/5\\
\end{tabular}


\end{document}

