\documentclass[11pt,a4paper]{article}
\usepackage{a4wide}
\usepackage{amsfonts}

\begin{document}

\title{On Higher Dimensional Interlacing Fibonacci Sequences, Continued
Fractions and Chebyshev Polynomials}
\author{Matthew Lettington (Cardiff)}
\date{}


\maketitle

\thispagestyle{empty}

\noindent
{\bf Abstract.}
We study higher-dimensional interlacing Fibonacci sequences and
their corresponding multi-dimensional continued fractions, generated
via both Chebyshev type functions and $m$-dimensional recurrence
relations. For each integer $m$, there exist both rational and
integer versions of these sequences, where the underlying $p$-adic
structure of the rational sequence enables the integer sequence to
be recovered. In particular, for the positive index sequences, one
can clear fractions if one know the number of prime divisors of $2m
+1$; in the negative index case the ``excess'' prime factors can be
removed using Weisman's congruence. When $2m+1$ is a prime these two
processes come into alignment.

From either the rational or the integer sequences we can construct a
continued fraction vector in $\mathbb{Q}^m$, which converges to an
irrational algebraic point in $\mathbb{R}^m$. The sequence terms can
be expressed as simple recurrences, trigonometric sums, binomial
polynomials and as sums over ratios of powers of the diagonals of
the regular unit $n$-gon. These sequences also exhibit a ``rainbow
type'' quality, corresponding to the Fleck numbers at negative
indices and the $m$-dimensional Fibonacci numbers at positive indices.

It is shown that the families of orthogonal generating polynomials
defining the recurrence relations employed, are divisible by the
minimal polynomials of certain algebraic numbers, and the three-term
recurrences and differential equations for these polynomials are
derived. Further results relating to the Christoffel-Darboux
formula, Rodrigues' formula and raising and lowering operators are
also discussed   Furthermore, it is shown that the Mellin transforms
of these polynomials satisfy a functional equation of the form
$p_n(s)=\pm p_n(1-s)$, and have zeros only on the critical line Re
$s=1/2$.

This is joint with M. W Coffey, J. L Hindmarsh, and J. Pryce.







\end{document} 