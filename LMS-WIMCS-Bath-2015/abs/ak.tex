\documentclass[11pt,a4paper]{article}
\usepackage{a4wide}
\usepackage{amsfonts}

\begin{document}

\title{Mathematical modelling of scattering in presence of metamaterial objects}
\author{Alexander Kiselev (L'viv, Ukraine)}
\date{}


\maketitle

\thispagestyle{empty}

\noindent
{\bf Abstract.}
Recently, there has been a considerable interest in considering materials with negative permittivity and/or permeability (so-called metamaterials). In particular, the argument has been made that there is a possibility of cloaking (perfect or not) in presence of so-called superlenses, i.e., lenses partially made of metamaterials.

In this talk, based on joint research with Dr Kirill Cherednichenko, I will try to cover the models considered so far and the corresponding results as well as to present our own recent results related to the rigorous construction of scattering matrix in the low-energy limit. The approach utilized by us allows not only to consider a superlens of an arbitrary geometry, but also sheds light on the origin and particulars of the process of cloaking. In particular, we will argue that there exists a "resonance" problem on the compact scatterer which spectral properties give rise to cloaking.




\end{document} 