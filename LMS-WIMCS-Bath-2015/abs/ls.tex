\documentclass[11pt,a4paper]{article}
\usepackage{a4wide}
\usepackage{amsfonts}

\begin{document}

\title{Homogenization of dislocation dynamics}
\author{Lucia Scardia (Bath)}
\date{}


\maketitle

\thispagestyle{empty}

\noindent
{\bf Abstract.}
It is well known that the plastic, or permanent, deformation of a
metal is caused by the movement of curve-like defects in its crystal
lattice. These defects are called dislocations. What is not known is
how to use this microscale information to make theoretical predictions
at the continuum scale. A mathematical procedure that has proved to be
very successful for the micro-to-macro upscaling of equilibrium
problems in materials science is Gamma-convergence.

Macroscopic plasticity, however, is heavily dependent on dynamic
properties of the dislocation curves. Motivated by this, M.G. Mora,
M.A. Peletier and I recently upscaled a time-dependent system of
discrete, interacting dislocations by combining Gamma-convergence
methods with the theory of rate-independent systems. In the continuum
limit we obtained an evolution law for the dislocation density. In
this talk I will present this result and discuss its limitations and
further extensions towards more realistic and complex systems.







\end{document} 