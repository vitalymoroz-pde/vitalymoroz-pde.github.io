\documentclass[11pt,a4paper]{article}
\usepackage{a4wide}
\begin{document}

\title{Wiener-Hopf analysis in the dynamic fracture of periodic flexural structures}
\author{Michael Nieves (Liverpool)}
\date{}


\maketitle

\thispagestyle{empty}

\noindent
{\bf Abstract.}
In this talk, we consider the dynamic fracture problem for a discrete lattice strip composed of periodically placed point masses connected by beams. A discontinuity representing the fracture is assumed to propagate steadily along the strip. It is shown that the governing equations for the masses form a discrete system of second order ODEs involving the displacements and rotations of the masses. After introducing a moving coordinate system which follows the front of the discontinuity and applying the Fourier transform, the problem is reduced to a Wiener-Hopf equation posed along the line involving the discontinuity. The kernel function inside this equation is analysed to obtain the dispersion properties of the strip and from this the complete solution is obtained through the Cauchy factorisation of the kernel function. Using the singularities of the kernel function, we then analyse the existence of reflected and transmitted waves caused by the interaction of the fracture front with waves generated by a remote load. Hence the distribution of the energy supplied by the load inside the structure can be considered and the dependency of the criterion for steady-state fracture on fracture speed is found. Numerical simulations are given which show that the model can be used to predict the speed of advancement of the fracture through a structure subjected to an oscillating point force. The fracture speed's dependence on the point force amplitude is also presented.



\end{document} 