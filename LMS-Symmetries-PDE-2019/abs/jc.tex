\documentclass[11pt,a4paper]{article}
\usepackage{a4wide}
\usepackage{amsfonts}

\begin{document}

\title{Nonlinear Aggregation-Diffusion Equations in the Diffusion-dominated and Fair competitions regimes}
\author{J.A. Carrillo (Imperial College London)}
\date{}


\maketitle

\thispagestyle{empty}

\noindent
{\bf Abstract.}
We analyse under which conditions equilibration between two competing effects, repulsion modelled by nonlinear diffusion and attraction modelled by nonlocal interaction, occurs. I will discuss several regimes that appear in aggregation diffusion problems with homogeneous kernels. I will first concentrate in the fair competition case distinguishing among porous medium like cases and fast diffusion like ones. I will discuss the main qualitative properties in terms of stationary states and minimizers of the free energies. In particular, all the porous medium cases are critical while the fast diffusion are not. In the second part, I will discuss the diffusion dominated case in which this balance leads to continuous compactly supported radially decreasing equilibrium configurations for all masses. All stationary states with suitable regularity are shown to be radially symmetric by means of continuous Steiner symmetrisation techniques. Calculus of variations tools allow us to show the existence of global minimizers among these equilibria. Finally, in the particular case of Newtonian interaction in two dimensions they lead to uniqueness of equilibria for any given mass up to translation and to the convergence of solutions of the associated nonlinear aggregation-diffusion equations towards this unique equilibrium profile up to translations as time tends to infinity. This talk is based on works in collaboration with S. Hittmeir, B. Volzone and Y. Yao and with V. Calvez and F. Hoffmann.

\end{document} 