\documentclass[11pt,a4paper]{article}
\usepackage{a4wide}
\usepackage{amsfonts}

\begin{document}

\title{Symmetry of constrained minimizers of a Cahn-Hilliard energy on the torus}
\author{Alfred Wagner (Aachen)}
\date{}


\maketitle

\thispagestyle{empty}

\noindent
{\bf Abstract.} 
We are interested in symmetry of constrained minimizers of a Cahn-Hilliard  energy on the
torus. Steiner symmetrization is a natural tool in such a setting, and it is easy to use Steiner
symmetrization to show that there exist minimizers with
the symmetries of the torus. In this paper, we show that in fact any  constrained
minimizer is (up to a shift) equal to its Steiner symmetrization. 
To do so, we formulate general sufficient conditions
for a function on the torus to be equal to
its Steiner symmetrization. Applying the result to our Cahn-Hilliard model, we obtain in particular 
that the superlevel sets of minimizers are simply connected.
In two dimensions, we use this together with the Bonnesen inequality to derive a new bound
on the sphericity of minimizers, which rules out
phenomena such as "tentacles.''
\newline
An even simpler rearrangement is the two-point rearrangement or polarization of a function. In 
general two-point rearrangements give weaker results than symmetrization. For the Cahn-Hilliard 
problem, however, we will obtain from two-point rearrangements that a minimizer is equal to its 
reflection with respect to some hyperplane and from here deduce strict monotonicity properties.
\newline
\newline
This is joint work with Maria G. Westdickenberg (RWTH Aachen University) and Michael Gelantalis, 
(University of Tennessee).


\end{document} 