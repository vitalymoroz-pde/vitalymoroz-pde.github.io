\documentclass[11pt,a4paper]{article}
\usepackage{a4wide}
\usepackage{amsfonts}

\begin{document}

\title{Logarithmic Sobolev inequalities with monomial weights}
\author{Filomena Feo (Naples)}
%Dipartimento di Ingegneria, Universit\`{a} degli Studi di Napoli "Parthenope",
\date{}


\maketitle

\thispagestyle{empty}

\noindent
{\bf Abstract.}
It is well-known that Euclidean Logarithmic Sobolev inequalities can be derived from Sobolev one (see \cite{Beckner-Pearson}). Following this idea, we derive some $L^p$ (Euclidean) Logarithmic Sobolev inequality with the monomial weights starting from a suitable weighted Sobolev inequality, which is contained in \cite{Cabre-RosOton}.
In this framework the product structure of both the Euclidean space and the weight is essential. When $p=2$ the obtained inequality is sharp. Indeed in the used method the asymptotic behaviour of the involved constant is essential.
Some applications and some derived inequalities are also considered.
\\
This talk is based on a joint work with F. Takahashi.

\small
\begin{thebibliography}{99}

\bibitem{Beckner-Pearson}
W. Beckner, and M. Pearson:
{\it On sharp Sobolev embedding and the logarithmic Sobolev inequalitiy},
\newblock Bull. London Math. Soc., \textbf{30}  (1998), 80-84.

\bibitem{Cabre-RosOton}
X. Cabr\'e, and X. Ros-Oton:
{\it Sobolev and isoperimetric inequalities with monomial weights},
\newblock J. Differential Equations, \textbf{255}  (2013), 4312-4336.
\end{thebibliography}

\end{document} 