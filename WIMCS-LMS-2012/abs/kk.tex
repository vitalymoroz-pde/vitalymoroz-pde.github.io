\documentclass[11pt,a4paper]{article}
\usepackage{a4wide}
\begin{document}

\title{Quasiconvexity at the boundary and the nucleation of austenite}
\author{Konstantinos Koumatos (Oxford, UK)}
\date{}

\maketitle

\thispagestyle{empty}

\noindent
{\bf Abstract.}
In a remarkable experiment of Hanu� Seiner, the high temperature phase (austenite) of a CuAlNi shape-memory alloy is nucleated in the low temperature phase (martensite) by localized heating. It is observed that, regardless of where the localized heating is applied, the nucleation points are always located at one of the corners of the specimen - a rectangular bar in the austenite.


In a simplified model, we propose an explanation for the location of the nucleation points by showing that the martensite is a minimizer of the energy with respect to localized variations in the interior, on faces and edges of the sample but not at all corners, where a localized microstructure can lower the energy. The stability of the martensite is established by showing that the free-energy function satisfies a set of quasiconvexity conditions at the martensite in the interior, on faces and edges, provided the sample is suitably cut. This is joint work with John Ball.




\end{document} 