\documentclass[12pt,a4paper]{article}
\begin{document}

\title{Integral stability for Calder\'{o}n Inverse Problem}
\author{Daniel Faraco}
\date{}

\maketitle

\thispagestyle{empty}

\begin{abstract}
Calder\'{o}n inverse problem consists on the determination of the coefficient (the
conductivity) of an isotropic elliptic equation in divergence form the Dirichlet
to Neumann map (boundary measurements). The problem in the plane was
solved in its full generality by K.Astala and L.P\"{a}iv\"{a}rinta in 2006 by means of
quasiconformal mappings. However the arguments in the proof are difficult to
quantify since they make strong use of the topological degree. Moreover existing
counterexamples shows that the process is not stable in general in absence of
continuity in the $L^{\infty}$ norm.

\smallskip
I will discuss a recent joint work with A.Ruiz (Madrid) and A.Clop (Barcelona)
were we show the stability of the process respect to the $L^2$ norm for coefficients
lying in fractional Sobolev spaces. In the way, we had to investigate how quasiconformal
mappings interact with these spaces, a topic of independent interest.



\end{abstract}

\end{document}