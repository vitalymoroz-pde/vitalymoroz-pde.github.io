\documentclass[11pt,a4paper]{article}
\usepackage{a4wide}
\begin{document}

\title{On the numerical relaxation of single-slip plasticity\\ in finite strains}
\author{Antonio Orlando (Swansea)}
\date{In collaboration with: Sergio Conti (Bonn) and\\ Carsten Carstensen (HU-Berlin)}

\maketitle

\thispagestyle{empty}

\noindent
{\bf Abstract.}
The modeling of the elastoplastic behavior of single crystals
with infinite latent hardening 
leads to a nonconvex energy density, whose
minimization produces fine structures. The effective macroscopic behaviour 
can be characterised by means of the quasiconvex
envelope of the energy density; unfortunately a closed form expression
for the latter is known only in few very  simplified cases. 
One is therefore lead to a new computational challenge, namely
numerical relaxation. This is  faced with huge numerical difficulties 
since it involves the minimization of a nonconvex function with clusters of local minima. 

\smallskip With the objective of gaining better insight in the type of
microstructure that can develop, and in the type of numerical
minimization algorithm that can be used for the relaxation, we study a
simplified model problem in two-dimensional, geometrically nonlinear
plasticity, with a single slip system and a linear hardening law. A
different  analysis of the relaxation of the same model was previously
given in \cite{BarCarHacHop04,MieLamGue04}. 

\smallskip By constraining the elastic part of
the deformation to be a rotation, we consider a first example where the dissipation contribution to the 
incremental energy is neglected, and a second one where the plastic free energy is neglected.
For both cases,
the quasiconvexification of the energy density can be determined in closed form \cite{ConPre,ConThe05}.

\smallskip More refined models are then obtained by assuming the microstructure to
have the  form of a  laminate of second order which 
is supported either on rigid-plastic deformations, or on purely elastic ones,
or on a mixture of purely elastic and plastic ones.
In all these cases the relaxation can be reduced to the minimization of a
function of only one variable.

\smallskip We use the above results for 
the numerical minimization of the full energy density, including
dissipation, and removing the kinematic constraint.
We then assess the precision of our relaxation by determining at each 
macroscopic strain a polyaffine function which coincides with the unrelaxed energy on the 
support of the laminate and checking that
it is below the condensed energy, up to a very small error.

\smallskip We conclude with some numerical examples and comparisons with the 
literature \cite{BarCarHacHop04,MieLamGue04,CarConOrl08}.



{\small
\begin{thebibliography}{99} 

\bibitem{BarCarHacHop04} 
        Bartels S., Carstensen C, Hackl K., Hoppe U., 
        \textit{Effective relaxation for microstructure simulations: 
	algorithms and applications}, Comput. Meth. Appl.
        Mech. Engng. 193, 5143-5175, (2004)   

\bibitem{ConPre} 
        Conti S., 
        \textit{Relaxation of single-slip single-crystal plasticity with 
        linear self-hardening}, In: Proceedings of the 3rd Conference on 
	`Multiscale Materials Modeling', P. Gumbsch (Ed.), Freiburg, 2006, 81-85 

\bibitem{ConThe05}
	Conti S., Theil F.,
	\textit{Single-slip elastoplastic microstructures},
	Arch. Rational Mech. Anal. 178, 125-148, (2005)

\bibitem{MieLamGue04}
	Miehe C., Lambrecht M., G\"{u}rses E., 
	\textit{Analysis of material instabilities in inelastic solids by incremental energy minimization
	and relaxation methods: evolving deformation microstructures in finite plasticity},
	J. Mech. Phy. Solids 52, 2725-2769 (2004)

\bibitem{CarConOrl08}
        Carstensen C., Conti S., Orlando A.,
        \textit{Mixed analytical-numerical relaxation in single-slip crystal plasticity},
        Accepted for publication in Cont. Mech. \& Thermod. 

\end{thebibliography}











\end{document}